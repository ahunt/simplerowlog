\documentclass[11pt,a4paper]{book}
\usepackage{html}
%\usepackage[pdftex]{graphicx}
\hfuzz = .6pt
\hypersetup{colorlinks, 
           citecolor=black,
           filecolor=black,
           linkcolor=black,
           urlcolor=black,
           bookmarksopen=true}
%\usepackage[colorlinks]{hyperref}
  \newcommand{\srl}{\textit{simple rowLog }} 
  \newcommand{\code}[1]{\texttt{#1}}
  \newcommand{\link}[2]{\htmladdnormallinkfoot{#1}{#2}}
  \newcommand{\command}[1]{\begin{quote}\begin{normalsize}\code{#1}\end{normalsize}\end{quote}}
\title{simple rowLog Manual}
\author{Andrzej JR Hunt}
\date{version 0.1 -- 24. October 2009}
\begin{document}
\maketitle

% Foreword and licensing info
\begin{footnotesize}
\begin{verbatim}
Copyright (C)  2009  Andrzej JR Hunt
Permission is granted to copy, distribute and/or modify this document
under the terms of the GNU Free Documentation License, Version 1.3
or any later version published by the Free Software Foundation;
with no Invariant Sections, no Front-Cover Texts, and no Back-Cover Texts.
A copy of the license is included in the section entitled "GNU
Free Documentation License".
\end{verbatim}\end{footnotesize}
\html{\hrule} % hline only in html
\srl is a simple rowing logbook for use in rowing clubs. It is intended to allow easy organisation of log data from outings and other happenings. This manual is intended to help in the use of simple rowLog, and to document the features it includes. \srl should be simple enough to use without referring to this manual, however it is highly reccommended that administrators read the manual to better know how to operate the software.
\html{\hrule}\latex{\newline} % hline for html, newline (parbreak) for normal
This manual is available in its latest version in \link{html}{http://srl.ahunt.org/manual/current/html/} and \link{pdf}{http://srl.ahunt.org/manual/current/manual.pdf} format online in the latest version, as well as included with distribution packaged versions of simple rowLog. It is will also be viewable directly in simple rowLog through use of the \code{Help} menu, and then selecting \code{Help contents}. [Unimplemented as yet.]
\
\
% TOC
\tableofcontents

% General chapter
\chapter{General usage}
% Administration chapter
\chapter{Administration}
\section{User management}
\srl by default contains two users, \code{default} and \code{deleted}.
\subsection{To add a user}
In the admin dialog do bla.
\subsection{To delete a user}
All mentions of user in outings will be replaced with the deleted user.
\section{Boat management}
Easy peasy


% Apendix begins
\appendix

% The GNU FDL
\chapter{License}
This document is Copyright \copyright\ 2009 Andrzej JR Hunt. It is released under the terms of the GNU Free documentation license.
\section{GNU Free Documentation License}
\input{gnu-fdl-1.3}

% How to get the sources.
\chapter{Getting sources (simple rowLog and manual)}
The sources for \srl and this manual are kept on the \srl svn repository. You download the current development branch of simple rowlog using the commands:
\command{svn co svn://svn.savannah.nongnu.org/simplerowlog/trunk}
or
\command{svn co http://svn.savannah.nongnu.org/simplerowlog/trunk}
(The second method pulls the sources using http which is slower.)
It is also possible to \link{browse the sources online}{http://svn.savannah.nongnu.org/viewcvs/?root=simplerowlog}.


The document sources are found in the directory \code{doc/src}. The program sources are in \code{src}. Note that if you download the sources from svn, external sources aren't included, but can be downloaded using:\command{ant getdep-src}which downloads them into \code{src/external}.


\end{document}
